\documentclass[10pt,a4paper]{article}
%\usepackage[table]{xcolor}
%\usepackage{float}
\usepackage[spanish]{babel}
\usepackage{amsmath}
%\usepackage{amssymb}
\usepackage{graphicx}
%\usepackage{amsfonts}
\usepackage[utf8]{inputenx}
\usepackage{pdfpages}
\usepackage{listings}
%\usepackage{algorithm2e}
%\usepackage{listings}
%\usepackage{pdfpages}
%\usepackage{tabularx}
%\usepackage{color}
\usepackage{anysize}
\usepackage{fancyhdr}
\usepackage{ulem}
\usepackage{hyperref}
%\usepackage{caption}
\usepackage[font=footnotesize]{caption}
\definecolor{deepblue}{RGB}{0,0,153}
\definecolor{deepred}{RGB}{153,0,0}
\definecolor{deepgreen}{RGB}{51,102,0}
\definecolor{deepyellow}{RGB}{204,204,0}
\marginsize{2cm}{2cm}{1cm}{1.5cm} % depende de anysize
%\renewcommand*{\thefootnote}{\Roman{footnote}}
%\usepackage{hyperref}
%\hypersetup{
%    colorlinks=true,
%    citecolor=black,
%    filecolor=black,
%    linkcolor=black,
%    urlcolor=black,
%    linktoc=all
%}



%\title{Multímetros en Corriente Continua}
\title{Universidad de Buenos Aires - FIUBA \\
		66.20 Organización de Computadoras \\
		Trabajo Práctico 1: Assembly Mips\\}
\author{	Joaquin Segui, \textit{Padrón Nro. 91.451}                     \\
            \texttt{ segui.joaquin@gmail.com }                                          \\[2.5ex]
            Pernin Alejandro, \textit{Padrón Nro. 92.216}                     \\
            \texttt{ale.pernin@gmail.com}                                              \\[2.5ex]
            Menniti Sebastián Ezequiel, \textit{Padrón Nro. 93.445}                     \\
            \texttt{ mennitise@gmail.com }                                              \\[2.5ex]}
\date{}

%\lfoot{asdasd}


\begin{document}
%\includepdf{attachments/caratula.pdf}


%\newpage\null\thispagestyle{empty}\newpage
\maketitle\thispagestyle{empty}

\newpage\null\thispagestyle{empty}%\newpage

%\newpage
%\tableofcontents

%\newpage\null\thispagestyle{empty}\newpage

\section{Introducción}

Se implemento un programa, en lenguaje C, que se encarga de multiplicar matrices de números reales, representados en punto flotante de doble precisión. A diferencia del trabajo anterior, la función encargada de la multiplicación de las matrices se encuentra escrita en el lenguaje Assembly Mips.


\section{Diseño e Implementación}

Dada las limitaciones del assembly Mips, fue necesario realizar algunos cambios respecto de la entrega anterior

\begin{itemize}
	\item Representar las matrices como un unico arreglo de doubles.
	\item Diseñar el manejo de los argumentos mediante el stack.
\end{itemize}

Para manejar diversas variables dentro del programa, se utilizo un Stack Frame de 32 bytes.

	\begin{center}
			{\footnotesize \begin{tabular}{ |l|c| }

			\hline
				52 & columnas 2 \\ \hline
				48 & columnas 1 \\ \hline	
				44 & a3 (filas 1) \\ \hline
				40 & a2 (double* src1) \\ \hline
				36 & a1 (double* src2) \\ \hline
				32 & a0 (double* dest) \\ \hline
				28 & ra \\ \hline
				24 & fp \\ \hline
				20 & gp \\ \hline
				16 & i \\ \hline
				12 & j \\ \hline
				8 & k \\ \hline
				4 & ra (not used) \\ \hline
				0 & accum \\ \hline
				
			\end{tabular}}\captionof{table}{Stack Frame}\label{tab:regtension}
	\end{center}

Alli se alojan los valores empleados para los controles de los ciclos y el acumulador auxiliar para la multiplicación. Si bien la posición 4 no se utiliza, el frame debe ser multiplo de 8 bytes y por ello se mantienen los 32 bytes.

\section{Comandos para compilar el programa}
	\subsection{Copiar a VM}
		Dado que el programa se corre dentro de la maquina virtual gxemul, se facilita un script en bash que copia el contenido de la carpeta en el guest (considerando la configuración default vista en clase). Para ello invocar desde el host \texttt{copiar\_tp.sh} el mismo se copiará en \textit{/root/tp1}.

	\subsection{Compilacion y Ejecucion}
		Para facilitar la compilacion se utiliza un \textit{Makefile}, para invocar la compilación del programa ejecutar desde una consola, dentro del mismo directorio que el código fuente: \texttt{make}.

		Asimismo se proviciona un script que realiza pruebas con un set de datos preexistente, para invocarlo: \texttt{bash pruebas\_mips.sh}.

\section{Pruebas}
	En esta sección se detallarán las pruebas realizadas. Los archivos utilizados se encuentran el el directorio \textit{test\_files}.
	\subsection{Casos Exitosos}
		Los casos exitosos están comprendidos por los set de datos \textit{test1} y \	textit{test2}.	
		En el caso del	 primero:\	
	
		\lstinputlisting[basicstyle=\footnotesize]{../src/test_files/test1.txt}
	
		Representa la operación
		\begin{center}
		$\begin{pmatrix}
		1 & 2
		\end{pmatrix}
		*
		\begin{pmatrix}
		1 & 0 & 4 \\ 5 & 1 & 3
		\end{pmatrix}
		=
		\begin{pmatrix}
		11 & 2 & 10
		\end{pmatrix}
		$
		\end{center}
	
		cuya salida por consola mediante el script de pruebas es
	
		\texttt{1x3 11 2 10 }
	
		como es esperable.\\
	
		El segundo set de datos es:
		\lstinputlisting[basicstyle=\footnotesize]{../src/test_files/test2.txt}
	
		representando las operaciones
	
		\begin{center}
		$\begin{pmatrix}
		1 \\ 2 \\ 3
		\end{pmatrix}
		*
		\begin{pmatrix}
		0 & 3 & 1
		\end{pmatrix}
		=
		\begin{pmatrix}
		0 & 3 & 1 \\ 0 & 6 & 2 \\ 0 & 9 & 3
		\end{pmatrix}
		$\end{center}
	
		\begin{center}
		$\begin{pmatrix}
		1 & 3
		\end{pmatrix}
		*
		\begin{pmatrix}
		1 & 0 & 4 \\ 5 & 1 & 0
		\end{pmatrix}
		=
		\begin{pmatrix}
		16 & 3 & 4
		\end{pmatrix}
		$\end{center}
	
		cuya salida se obtuvo correctamente
	
		\texttt{3x3 2.22507e-308 3 1 3.33761e-308 6 2 4.45015e-308 9 3 }

		\texttt{1x3 16 3 4}

		\textbf{Nota:} Queda a determinar el causal de que el 0 (cero) no se encuentre expresado exactamente.

	\subsection{Casos de error}
		Como casos de error del programa probamos matrices incompatibles para su multiplicacion y matrices mal definidas.

		Uno de estos casos es el de tener dos matrices cuyas dimensiones hacen incompatibles la multiplicación entre sí. Este es el caso del set \textit{test3}.

		\lstinputlisting[basicstyle=\footnotesize]{../src/test_files/test3.txt}
		\begin{center}
		$\begin{pmatrix}
		1 & 2
		\end{pmatrix}
		*
		\begin{pmatrix}
		1 & 0 & 4
		\end{pmatrix}
		$\end{center}

		Al ejecutar dicha prueba, el programa termina con el siguiente mensaje:

		\texttt{Dimensiones no compatibles para multiplicar}\\

		Otra prueba es tener una cantidad impar de matrices, por lo cuál una no podrá ser multiplicada. Por ejemplo \textit{test4}.

		\lstinputlisting[basicstyle=\footnotesize]{../src/test_files/test4.txt}
		
		Como resultado arroja

		\texttt{3x3 2.22507e-308 3 1 3.33761e-308 6 2 4.45015e-308 9 3 }

		\texttt{Fallo al leer dimensiones}\\

		Otros casos de prueba, consisten en definir dimensiones de matrices inconsistentes con la cantidad de elementos leidos. Al ver \textit{test5}

		\lstinputlisting[basicstyle=\footnotesize]{../src/test_files/test5.txt}

		\begin{center}
		$\begin{pmatrix}
		1 \\ 2 \\ X
		\end{pmatrix}
		*
		\begin{pmatrix}
		0 & 3 & 1
		\end{pmatrix}
		$\end{center}

		si bien las dimensiones declaradas son compatibles para su multiplicación, los elementos provistos son inconsistentes. Dicha prueba arroja:

		\texttt{Cantidad elementos distinta a dimensiones de matriz}


\newpage
\section{Codigo fuente del programa}

\subsection{En lenguaje C}

\lstinputlisting[
			language=C,
			basicstyle=\footnotesize,
			numbers=left,
			stepnumber=1,
			numbersep=4pt,
			tabsize=2,
			otherkeywords={self}, 
			keywordstyle=\color{deepred},
			stringstyle=\color{deepgreen},
			commentstyle=\color{deepblue},
			]{../src/main.c}

\newpage
\subsection{Funcion en MIPS32}

\lstinputlisting[
			language=Assembler,
			basicstyle=\footnotesize,
			numbers=left,
			stepnumber=1,
			numbersep=4pt,
			tabsize=2,
			otherkeywords={self}, 
			keywordstyle=\color{deepred},
			stringstyle=\color{deepgreen},
			commentstyle=\color{deepblue},
			]{../src/myMultiplicar.S}

\newpage\thispagestyle{empty}
\includepdf[pages=1,pagecommand=\section{Enunciado}]{tp1-2015-2q.pdf}
\includepdf[pages=2-,]{tp1-2015-2q.pdf}

\end{document}