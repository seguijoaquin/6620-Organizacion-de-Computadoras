\documentclass[a4paper,10pt]{article}

\usepackage{graphicx}
\usepackage[ansinew]{inputenc}
\usepackage[spanish]{babel}

\title{		\textbf{Trabajo Pr�ctico Nro. 0}}

\author{	Joaquin Segui, \textit{Padr�n Nro. 91.451}                     \\
            \texttt{ segui.joaquin@gmail.com }                                          \\[2.5ex]
            Pernin Alejandro, \textit{Padr�n Nro. 92.216}                     \\
            \texttt{ ale.pernin@gmail.com }                                              \\[2.5ex]
            Menniti Sebasti�n Ezequiel, \textit{Padr�n Nro. 93.445}                     \\
            \texttt{ mennitise@gmail.com }                                              \\[2.5ex]
            \normalsize{2do. Cuatrimestre de 2015}                                      \\
            \normalsize{66.20 Organizaci�n de Computadoras  $-$ Pr�ctica Martes}  \\
            \normalsize{Facultad de Ingenier�a, Universidad de Buenos Aires}            \\
       }
\date{}

\begin{document}

\maketitle
\thispagestyle{empty}   % quita el n�mero en la primer p�gina


\section{Introducci�n}

Se implemento un programa, en lenguaje C, que se encarga de multiplicar matrices de n�meros reales, representados en punto flotante de doble precisi�n.


\section{Dise�o e Implementaci�n}

Las matrices a multiplicar se ingresan por entrada est�ndar (\texttt{stdin}), donde cada linea representada una matr�z completa en formato de texto, describiendola mediante el siguiente formato:

\begin{center}

\textit{  NxM a_{1,1}  a_{1,2}  ...  a_{1,M}  a_{2,1}  a_{2,2}  ...  a_{2,M}  ...  a_{N,1}  a_{N,2}  ...  a_{N,M}  }

\end{center}

Esta linea representa a una matr�z \textit{A}, donde \textit{N} es la cantidad de filas y \textit{M} la cantidad de columnas de la matr�z \textit{A}. Los elementos de la matr�z \textit{A} son los \texit{ a_{x,y} } , donde \textit{x} e \textit{y} son los indices de fila y columna respectivamente. El fin de linea se delimita con el caracter \textit{newline}.

Por cada par de matrices que se presentan en la entrada, el programa en primer lugar, se encarga de cargarlas, luego verifica que las matrices cumplan con la condici�n para que la multiplicaci�n sea posible (Se verifica que la cantidad de columnas de la primer matr�z sea igual a la cantidad de filas de la segunda matr�z), y en el caso que la cumplan, se procede a multiplicarlas. El resultado obtenido lo presenta por salida est�ndar (\texttt{stdout}), en el mismo formato mencionado anteriormente. Este proceso se repite hasta que llegue al final del archivo de entrada (\texttt{EOF}). Si se encontrara con un error, el programa lo informa por \texttt{stderr} y se detiene su ejecuci�n.



\section{Comandos para compilar el programa}

...

\section{Pruebas}

...

\section{Codigo fuente del programa}

\subsection{En lenguaje C}

\texttt{
... (Codigo en C)
}

\subsection{Codigo MIPS32 generado por el compilador}

\texttt{
... (Codigo de MIPS32)
}

\section{Enunciado}

Ac� ir�a el enunciado, no se si se anexa aparte o lo agregamos ac�


\section{Conclusiones}

...  Conclusiones  ...

\end{document}
